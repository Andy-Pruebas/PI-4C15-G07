\chapter{Introducción}
En la actualidad gran porcentaje de la población peruana están desinformadas sobre temas de suma importancia. Esto involucra los temas de enfermedades provocando que la población no sepa actuar de manera preventiva frente a una enfermedad ni cómo identificarla. La anemia es una de las principales enfermedades que sufren los menores de 6 a 59 meses de edad, siendo un alarmante debido a que un 32,8\% de niños padecen anemia, manteniéndose así aproximadamente desde el 2012 hasta 2018 que se realizó un último estudio de parte de organización panamericana de salud. Otra enfermedad que provoca pánico entre los peruanos es la diabetes siendo esta la quinta causa de muerte en el Perú aproximadamente a fines de 2018 se reportaron un millón de personas con esta enfermedad. Pero gran parte de las provincias no están informadas acerca de estos temas por el cual se presenta una aplicación que hace realiza una encuesta al usuario en el lenguaje que este prefiera: castellano, quechua y aimara para luego continuar a informarle al usuario sobre una de ambas enfermedades en el lenguaje que selecciono. 
\section{Propósito}
El propósito de este proyecto es el de realizar una aplicación que pueda informar a las personas que hablan Castellano, Quechua, acerca de enfermedades relacionadas como Anemia, Diabetes, o incluso su estilo y que puedan mejorar su salud.  
\section{Alcance}
\begin{itemize}
\item Su alcance es para las personas que hablen las lenguas \\castellano, quechua.
\item Las personas que usan el sistema solo podrán marcar su respuesta en el cuestionario que se le brinda un resultado en base a los estudios de enfermedades como la anemia, diabetes.

\end{itemize}
\section{Definiciones, siglas y abreviaturas}

\section{Referencias}